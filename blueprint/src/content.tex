% In this file you should put the actual content of the blueprint.
% It will be used both by the web and the print version.
% It should *not* include the \begin{document}
%
% If you want to split the blueprint content into several files then
% the current file can be a simple sequence of \input. Otherwise It
% can start with a \section or \chapter for instance.

\chapter*{Introduction}
\addcontentsline{toc}{chapter}{\protect\numberline{}Introduction}
I would like to present here a short and elementary proof of the following theorem.

\begin{theorem}\label{thm:sum-three-squares}
If $m$ is a positive integer not of the form $4^{a}(8n+7)$, then $m$ is the sum of three squares.
\end{theorem}

\begin{proof}

We make use of an elegant method of Professor H. Davenport~\cite{Davenport1947} in the Geometry of Numbers.

Without loss of generality we will prove Theorem~\ref{thm:sum-three-squares} only when $m$ is square free. (In the following $m$ will be assumed to be square free.)
In Section~\ref{sec:case1} we shall prove Theorem~\ref{thm:sum-three-squares} when $m\equiv3$ (mod 8). In Section~\ref{sec:case2} we will merely outline the proof when $m\equiv1, 2, 5, 6$ (mod 8), as the proof is almost identical except for minor changes.

We shall only assume the reader is familiar with the elementary facts of the law of quadratic reciprocity, Minkowski's Theorem on lattice points contained within convex symmetric bodies; and when a positive integer is the sum of two squares.

\phantomsection
\refstepcounter{section}
\label{sec:case1}
\noindent 1. Let $m$ be a positive square free integer $\equiv3$ (mod 8), and $m=p_{1}p_{2}\cdots p_{r}$ where $p_{j}$'s are primes.
Denote by $q$ a positive prime which satisfies

\begin{align}
\label{eq:1} \tag{1}
(-2q/p_{j})&=+1, \quad j=1, 2, \cdots, r \\
\label{eq:2} \tag{2}
q&\equiv1 \pmod{4}
\end{align}

with $(a/b)$ denoting the Jacobi Symbol. We see that such a prime exists by Dirichlet's theorem regarding primes in an arithmetic progression,
as \eqref{eq:1} and \eqref{eq:2} merely necessitate that $q$ lie within certain relatively prime residue classes (mod $4m$).

By \eqref{eq:1} and \eqref{eq:2}
\begin{align*}
1&=\prod_{j=1}^{r}(-2q/p_{j})=\prod_{j=1}^{r}(-2/p_{j})(q/p_{j}) \\
\intertext{(3)}
&=(-2/m)\prod_{j=1}^{r}(p_{j}/q) \\
&=(-2/m)(m/q)=(-2/m)(-m/q) \\
&=(-m/q)
\end{align*}

as $m\equiv3$ (mod 8).
Hence, as $q$ is an odd prime we can find an odd integer $b$ such that $b^{2}\equiv-m$ (mod $q$), or
\begin{equation}
b^{2}-qh_{1}=-m. \label{eq:4} \tag{4}
\end{equation}
Considering \eqref{eq:4} (mod 4) yields $1-h_{1}\equiv+1$ (mod 4), or $h_{1}=4h$ where $h$ is a rational integer and
\begin{equation}
b^{2}-4qh=-m. \label{eq:5} \tag{5}
\end{equation}
Utilizing \eqref{eq:1} we can find an integer $t$ such that
\begin{equation}
t^{2}\equiv-1/2q \pmod{m}. \label{eq:6} \tag{6}
\end{equation}
We now consider the figure
\begin{equation}
R^{2}+S^{2}+T^{2}<2m \label{eq:7} \tag{7}
\end{equation}
where
\begin{align}
R&=2tqx+tby+mz, \notag \\
S&=(2q)^{1/2}x+b/(2q)^{1/2}y, \label{eq:8} \tag{8} \\
T&=m^{1/2}/(2q)^{1/2}y \notag
\end{align}

In the $(R, S, T)$ space \eqref{eq:7} defines a convex, symmetric (about the origin) body of volume $4/3\pi(2m)^{3/2}.$

The determinant of the transformations \eqref{eq:8} is $m^{3/2}$. Hence, in the $(x, y, z)$ space, \eqref{eq:7} represents a
convex symmetric body of volume $1/3(2^{7/2}\pi)$, and certainly $1/3(2^{7/2}\pi) > 8$.

Hence, by Minkowski's Theorem on convex symmetric body in three dimensions of volume $>8$, we know there exist
integer values of $x, y, z$ not all zero which satisfy \eqref{eq:7}. Let $x_{1}, y_{1}, z_{1}$ be the integers which satisfy
\eqref{eq:7} and \eqref{eq:8}, $R_{1}, S_{1}, T_{1}$ the corresponding values of $R, S, T$.

By \eqref{eq:8}

$$R_{1}^{2}+S_{1}^{2}+T_{1}^{2}=(2tqx_{1}+tby_{1}+mz_{1})^{2}$$
$$+((2q)^{1/2}x_{1}+b/(2q)^{1/2}y_{1})^{2}+(m^{1/2}/(2q)^{1/2}y_{1})^{2}$$
\begin{align}
&\equiv t^{2}(2qx_{1}+by_{1})^{2}+1/2q(2qx_{1}+by_{1})^{2} \notag \\
&\equiv0 \pmod{m} \label{eq:9} \tag{9}
\end{align}
by \eqref{eq:6}, the selection of $t$.
Furthermore,
\begin{align}
R_{1}^{2}+S_{1}^{2}+T_{1}^{2}&=R_{1}^{2}+((2q)^{1/2}x_{1}+b/(2q)^{1/2}y_{1})^{2}+(m^{1/2}/(2q)^{1/2}y_{1})^{2} \notag \\
&=R_{1}^{2}+1/2q(2qx_{1}+by_{1})^{2}+m/2qy_{1}^{2} \label{eq:10} \tag{10} \\
&=R_{1}^{2}+2(qx_{1}^{2}+bx_{1}y_{1}+hy_{1}^{2}). \notag
\end{align}
Let $v$ be the positive rational integer defined by
\begin{equation}
v=qx_{1}^{2}+bx_{1}y_{1}+hy_{1}. \label{eq:11} \tag{11}
\end{equation}
We note that $R_{1}$ is a rational integer and by \eqref{eq:9}, \eqref{eq:10}, and \eqref{eq:11} that $m|R_{1}^{2}+2v$, but by \eqref{eq:7} $R_{1}^{2}+2v<2m$. Furthermore $R_{1}^{2}+2v\ne0$, by the nondegenerate triangular transformation \eqref{eq:8} and the fact that not all $x_{1}, y_{1}, z_{1}$ equal zero. Hence,
\begin{equation}
R_{1}^{2}+2v=m. \label{eq:12} \tag{12}
\end{equation}
Let $p$ be an odd prime which exactly divides $v$ to an odd power, i.e. $p^{2n+1}||v$.
If $p$ does not divide $m$, then by \eqref{eq:12},
\begin{equation}
(m/p)=+1. \label{eq:13} \tag{13}
\end{equation}
By \eqref{eq:11}
\begin{equation}
4qv=(2qx_{1}+by_{1})^{2}+my_{1}. \label{eq:14} \tag{14}
\end{equation}
If $p|q$, then \eqref{eq:5}, $(-m/p)=1$.
If $p \nmid q$, then by \eqref{eq:14}
$$p^{2n+1}||e^{2}+mf^{2} \text{ or } (-m/p)=1.$$
Thus, in either case,
\begin{equation}
(-m/p)=+1 \label{eq:15} \tag{15}
\end{equation}
which combined with \eqref{eq:13} implies
\begin{equation}
(-1/p)=1 \text{ or } p\equiv1 \pmod{4}. \label{eq:16} \tag{16}
\end{equation}

If $p|v$, $p|m$, then by \eqref{eq:11} and \eqref{eq:12}

$$R_{1}^{2}+2v=m$$
or
\begin{equation}
R_{1}^{2}+\frac{1}{2q}((2qx_{1}+by_{1})^{2}+my_{1}^{2})=m \label{eq:17} \tag{17}
\end{equation}
which implies $p|R_{1}$, $p|(2qx_{1}+by_{1})$, and thus as $m$ is square free by dividing both sides of \eqref{eq:17} by $p$, yields
$$\frac{1}{2q}\frac{m}{p}y_{1}^{2}=\frac{m}{p} \pmod{p}$$
or
$$y_{1}^{2}\equiv2q \pmod{p}, \quad \left(\frac{2q}{p}\right)=+1$$
which combined with \eqref{eq:1} gives $(-1/p)=+1$ or $p\equiv1$ (mod 4).
Thus all odd primes which exactly divide $v$ to an odd power are $\equiv1$ (mod 4). Thus $2v$ is the sum of two square integers. By \eqref{eq:12} this implies $m$ is the sum of three square integers, which proves Theorem~\ref{thm:sum-three-squares} when $m\equiv3$ (mod 8).

\vspace{1em}

\phantomsection
\refstepcounter{section}
\label{sec:case2}
\noindent 2. If $m \equiv 1, 2, 5$ or $6$ (mod 8), we alter the proof in Section~\ref{sec:case1} in the following ways.
Let $q$ be a prime, $(-q/p_{j})=+1$ for all odd prime divisors of $m$, $q\equiv1$ (mod 4), and if $m$ is even,
$$m=2m_{1}, \quad (-2/q)=(-1)^{(m_{1}-1)/2}, \quad t^{2}\equiv-1/q \pmod{p_{j}},$$
$t$ odd, $b^{2}-qh=-m$
and
\begin{align*}
R&=tqX+tby+mz, \\
S&=q^{1/2}x+b/q^{1/2}y, \\
T&=m^{1/2}/q^{1/2}y.
\end{align*}
The proof will proceed exactly as in Section~\ref{sec:case1}, which will complete the proof when $m\equiv1, 2, 3, 5, 6$ (mod 8), and thus for all square free $m$.
\end{proof}
